% coding:utf-8

%----------------------------------------
%FOSADSVB, a LaTeX-Code for a summary of digital signal processing
%Copyright (C) 2015, Mario Felder & Michi Fallegger

%This program is free software; you can redistribute it and/or
%modify it under the terms of the GNU General Public License
%as published by the Free Software Foundation; either version 2
%of the License, or (at your option) any later version.

%This program is distributed in the hope that it will be useful,
%but WITHOUT ANY WARRANTY; without even the implied warranty of
%MERCHANTABILITY or FITNESS FOR A PARTICULAR PURPOSE.  See the
%GNU General Public License for more details.
%----------------------------------------

\chapter{Zufallssignale}
\section{Autokorrelation und Spektrum}
Mittelwert eines Zufallssignals $x[n]$ ist
\[ m_x = E\{x[n]\} \]
Autokorrelation
\[ \gamma_{xx}[m] = E\{x^*[n] \cdot x[n+m]\} \]
Wobei $\gamma_{xx}[0]$ die Leistung $P_x$ repräsentiert.\\\\
Für echte Zufallssignale gilt
\[ \gamma_{xx}[m] = \gamma_{xx}^*[-m] \]
Signale, welche den Mittelwert nicht bei Null haben, werden mit der Autokovarianz
charakterisiert
\[ c_{xx} = E\{ (x[n]-m_x)^*\cdot(x[n+m]-m_x) \} \]
\[ y_{xx} = c_{xx}[m] + _x^2 \]
Die Autokovarianz bei $m=0$ repräsentiert die Varianz
\[ E\{|x[n]-m_x|^2\} \]
Spektrum durch die DTFT im Intervall $[n_1,n_2]$
\[ X_{[n_1,n_2]}(\Omega) = \sum_{n=n_1}^{n_2}x[n]\cdot\e^{\im\Omega n} \]
Leistungsdichte Spektrum (power density spectrum)
\[ \Gamma_{xx}(\Omega) = \sum_{m=-\infty}^{\infty}\gamma_{xx}[m] \cdot
	\e^{-\im\Omega m} \]
~\\\\
Weisses Rauschen hat die Autokorrelation
\[ \gamma_{ww}[m] = \left\lbrace\begin{matrix}
	\sigma_w^2	& \textrm{if } m = 0\\
	0			& \textrm{if } m \neq 0
\end{matrix}\right. \]
Das Leistungsdichte Spektrum beim seissen Rauschen ist über $\Omega$ konstant.

%===============================================================================
\section{Spectral Shaping in LTI Systems}
Mittelwert des Ausgangssignals $y[n]$:
\[ m_y = H(0) \cdot m_x \]
mit
\[ H(\Omega) = \sum_{k=-\infty}^{\infty}h[k]\cdot\e^{\im\Omega k} \]
Autokorrelation von $y[n]$
\[ \gamma_{yy}[m] = h^*[-i]*\gamma_{xx}[m]*h[k] \]
Frequenzbereich
\[ \Gamma_{yy}(\Omega) = H^*(\Omega)\cdot\Gamma_{xx}(\Omega)\cdot H(\Omega) 
	= |H(\Omega)|^2 \cdot \Gamma_{xx}(\Omega) \]
	
%===============================================================================
\section{Lineare Modelle für Stochastische Prozesse}
Weisses Rauschen
\[ \Gamma_{ww}(z) = \sigma_w^2 \]
Ausgang des LTI
\[ \Gamma_{yy}(z) = \sigma_w^2 \cdot H(z) \cdot H(z^-1) \]
Weisses Rauschen kann erzeugt werden, indem ein bekanntes Zufallsignal mit
\[ \frac{1}{H(z)} \]
multipliziert wird. Ergibt für die Differenzengleichung
\[ w[n] = \frac{1}{b_0}\left( y[n] + \sum_{k=1}^{N}a_ky[n-k] - 
	\sum_{k=1}^{M}b_kw[n-k] \right) \]
	
%===============================================================================
\subsection{Moving average model}
$H(z)$ ist ein FIR Filter mit Ordnung $M$ welcher ein weisses Rauschen $w[n]$
in ein Zufallsignal transformiert:
\[ y[n] = \sum_{k=0}^{M}b_kw[n-k] \]
Im spezial Fall $b_0=b_1=\ldots=b_M=(M+1)^{-1}$ repräsentiert jedes Sample des
Filterausgans der Durchschnitt des Eingangsignals in einem verschiebbaren
Fenster der Länge $M+1$.\\
\\
Berechnung der Koeffizienten bei gegebener Autokorrelation $\gamma_{yy}$:
\[ \gamma_{yy}[m] = \left\lbrace \begin{matrix}
	\sigma_w^2\sum_{k=m}^{M}b_k\cdot b_{k-m}^* & \textrm{if } 0\leq m \leq M\\
	0	& \textrm{if } m > M\\
	\gamma_{yy}^*[-m]	& \textrm{if } m < 0
\end{matrix}\right. \]

%===============================================================================
\subsection{Autoregressive model}
Zufallssequenz ist durch einen All-pole Filter generiert:
\[ y[n] = w[n] - \sum_{k=1}^{N}a_ky[n-k] \]
Hat keine Pole und ist immer stabil.\\
Berechnung der Koeffizienten bei gegebener Autokorrelation $\gamma_{yy}$:
\[ \gamma_{yy}[m] = \left\lbrace \begin{matrix}
	-\sum_{k=1}^{N}a_k\cdot\gamma_{yy}[m-k] & \textrm{if } m>0\\
	\sigma_0^2-\sum_{k=1}^{N}a_k\cdot\gamma_{yy}[m-k] & \textrm{if } m =0\\
	\gamma_{yy}^*[-m]	& \textrm{if } m < 0
\end{matrix}\right. \]
Das Lösungsystem in Matrizenform:
\[ \begin{bmatrix}
	\gamma_{yy}[0] & \gamma_{yy}[-1] & \ldots & \gamma_{yy}[-N]\\
	\gamma_{yy}[1] & \gamma_{yy}[0] & \ldots & \gamma_{yy}[-N+1]\\
	\vdots & \vdots & & \vdots\\
	\gamma_{yy}[N] & \gamma_{yy}[N-1] & \ldots & \gamma_{yy}[0]
\end{bmatrix} \cdot \begin{bmatrix}
	1 \\ a_1 \\ \vdots \\ a_N \end{bmatrix} = \begin{bmatrix}
	\sigma_w^2 \\ 0 \\ \vdots \\ 0 \end{bmatrix} \]  
	
%===============================================================================
\subsection{ARMA model}
Ist die Zusammensetzung des MA und AR Models. Dabei sind die Ordnungen der
Polynomen $B(z)$ und $A(z)$ 1 oder höher. Somit hat der Pol-Nullstellenplan
Nullen und Pole. Dieser ist allerdings geeigneter als die Spezialfälle MA und
AR, da weniger Koeffizienten für ein gutes Model benötigt werden.