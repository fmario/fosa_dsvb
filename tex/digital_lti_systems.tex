% coding:utf-8

%----------------------------------------
%FOSADSVB, a LaTeX-Code for a summary of digital signal processing
%Copyright (C) 2019, Thomas Annen

%This program is free software; you can redistribute it and/or
%modify it under the terms of the GNU General Public License
%as published by the Free Software Foundation; either version 2
%of the License, or (at your option) any later version.

%This program is distributed in the hope that it will be useful,
%but WITHOUT ANY WARRANTY; without even the implied warranty of
%MERCHANTABILITY or FITNESS FOR A PARTICULAR PURPOSE.  See the
%GNU General Public License for more details.
%----------------------------------------
\chapter{Digitale LTI Systeme}

%===============================================================================
\section{Definition}
\textbf{Linearität:}
\[ x[n] = k_1 \cdot x_1[n] + k_2 \cdot x_2[n] \quad \Longrightarrow \quad y[n] = k_1 \cdot y_1[n] + k_2 \cdot y_2[n] \]
\textbf{Zeitinvarianz:}
\[ x[n] \rightarrow y[n] \quad \Longrightarrow \quad x[n-d] \rightarrow y[n-d] \]
\textbf{Erlaubte Operationen:}
\begin{itemize}[noitemsep,topsep=3pt]
	\item Multiplikation eines Signals mit einer Konstanten.
	\item Addition von zwei Signalen.
	\item Zeitverzögerung eines Signals um $k \cdot T_S$.
\end{itemize}

%===============================================================================
\section{Beschreibungsarten von LTI-Systemen}
Je nach Anwendung eignen sich unterschiedliche Beschreibungsarten, zwischen denen 
transformiert werden kann. 
\begin{table}[H]
\centering
\begin{tabular}{ccc}\hline
\multicolumn{1}{l}{} & \textbf{Impuls- oder Sprungantwort} ($h[n]$, $u[n]$) & Systemidentifikation, Messung \\
Zeitbereich & \textbf{Differenzengleichung} & Systemimplementation (Algorithmus) \\
 & \textbf{Signalflussdiagramm} & Systemimplementation (Architektur) \\ \hline
 & \textbf{Übertragungsfunktion} & Gekoppelte Analyse \& Implementation \\
Frequenzbereich & \textbf{Pol- Nullstellenplan} & Intuitives Analyse- \& Designvorgehen \\
\multicolumn{1}{l}{} & \textbf{Frequenzantwort} & Systemidentifikation, Analyse \& Design \\\hline 
\end{tabular}\vspace{-3mm}
\end{table}
\noindent \textbf{Nützliche Matlab-Befehle:}\\
\begin{tabular}{ll}
Definition: Impulsantwort mit $h[n] = \{1, -1\}$ & \verb|h = [1 -1];| \\
Plot: Impulsantwort mit Amplitude 1, Länge 10 & \verb|impz(h, 1, 10);| \\
Plot: Pol-Nullstellenplan & \verb|zplane(h,1);| \\
Plot: Frequenzgang (Gain \& Phase) & \verb|freqz(h,1,100); % freqz(b,a,N), b = Zähler, a = Nenner|
\end{tabular}\vspace{-3mm}

%===============================================================================
\section{Systembeschreibung im Zeitbereich}
\subsection{Impulsantwort / Faltung}
Das Ausgangssignal $y[n]$ kann durch Faltung des Eingangssignals $x[n]$ mit der Impulsantwort $h[n]$
ermittelt werden. 
\[ y[n] = \sum_{i=-\infty}^{\infty} x[i] \cdot h[n-i] = x[n] * h[n] \]
\subsection{Differenzengleichung}
Das $max(N,M)$ wird als \emph{Ordnung} des Systems bezeichnet. Falls $M \geq 1$ ist das System rekursiv, 
vergangene Ausgangswerte werden also in den aktuellen Ausgangswert miteinbezogen. $a_k$ \& 
$b_k$ beeinflussen das Systemverhalten.
\[y[n] = \sum_{k=0}^{N}b_k \cdot x[n-k]-\sum_{k=0}^{M} a_k \cdot y[n-k]\]
\subsection{Signalflussdiagramm}
Es existieren vier Grundlegende Varianten, welche jeweils Vor- und Nachteile für die 
praktische Implementation mit sich bringen. Im Kapitel "'Digitales Filterdesign"' sind
die vier Diagramme abgebildet. 

%===============================================================================
\section{Systembeschreibung im Frequenzbereich}
\subsection{Übertragungsfunktion}
Durch z-Transformation der Differenzengleichung ergibt sich:
\[y[n] = \sum_{k=0}^{N}b_k \cdot x[n-k]-\sum_{k=0}^{M} a_k \cdot y[n-k] \quad \laplace 
\quad Y(z) = \sum_{k=0}^{N}b_k z^{-k} X(z)-\sum_{k=0}^{M} a_k z^{-k} Y(z) \]
Durch Umstellen und Normieren von $a_0$ auf 1 erhält man: 
\[ H(z) = \frac{Y(z)}{X(z)} =
\frac{b_0+b_1 \cdot z^{-1} + b_2 \cdot z^{-2}+\ldots+b_N \cdot z^{-N}}
	{1+a_1 \cdot z^{-1} + a_2 \cdot z^{-2}+\ldots+a_M \cdot z^{-M}} \]
\subsection{Pol- Nullstellenplan}
Durch Faktorisieren der obigen Übertragungsfunktion $H(z)$ ergibt sich die folgende Darstellung:
\[  H(z) =  K_0 \cdot \frac{(z-z_1)(z-z_2)...(z-z_N)}
								{(z-p_1)(z-p_2)...(z-p_M)}
								\cdot z^{M-N} \]
\begin{itemize}[noitemsep,topsep=3pt]
	\item Für $N>M$ gibt es $N-M$ zusätzliche Polstellen bei $z=0$.
	\item Für $M>N$ gibt es $M-N$ zusätzliche Nullstellen bei $z=0$.
	\item Für kausale Systeme gilt, dass das System stabil ist, wenn: $\big| p_i \big| < 1,\  i =1,...,M$.
\end{itemize}					
\subsection{Frequenzgang}
Beschreibt das Systemverhalten als Funktion der Frequenz des Eingangssignals (frequency sweep). Es gilt: 
\[ h[n] \quad \laplace \quad H(\Omega) \]
In der Praxis wird die DTFT durch die DFT approximiert und als Polarkoordinaten dargestellt: 
\[ H(\Omega) = \big|H(\Omega) \big| \cdot \e^{j\varphi(H(\Omega))} \]	
\textbf{Amplitudengang:} Beschreibt die Verstärkung, die ein Signal der Frequenz $\Omega$ ausgesetzt ist:
\[ \big|H(\Omega)\big|_{dB} = 20 \cdot log_{10}(\big|H(\Omega)\big|) \]	
\textbf{Phasengang:} Beschreibt die Phasenverschiebung, der ein Signal der Frequenz $\Omega$ ausgesetzt ist:
\[ \varphi(Y(\Omega)) = \varphi(X(\Omega)) + \varphi(H(\Omega))\]	
\subsection{Beziehung zwischen Frequenzgang \& Übertragungsfunktion}
DTFT und $z$-Transformation sind mit folgender Beziehung verbunden:
\[ z = r \cdot e^{j\Omega} \]
Der Frequenzgang lässt sich also durch Betrachtung von $H(z)$ entlang des Einheitskreises erhalten.
\[ H(\Omega) = H(z)\big|_{z=e^{j\Omega}} \]	
\\
Durch Substitution von $z=e^{j\Omega}$ lässt sich  die Verstärkung \& Phase bei einer 
beliebigen Frequenz $\Omega$ berechnen.\\ 
\textbf{Amplitudengang:}
\[ \big| H(z)\big| =  |K| \cdot \frac{|(z-z_1)||(z-z_2)|..|(z-z_N)|}
								{|(z-p_1)||(z-p_2)|..|(z-p_M)|}
								\cdot |z|^{M-N} \]
\textbf{Phasengang:}
\[ \varphi(H(z)) = \sum_{k=1}^{N} \varphi(z-z_k) 
								-\sum_{k=1}^{M} \varphi(z-p_k)
								+\sum_{k=N+1}^{M} \varphi(z)  \]
~\\						
\emph{Trick 1 \& 2 auf der Folgeseite beachten, wenn Verstärkungen bei bestimmtem $\Omega$ berechnet werden müssen!}